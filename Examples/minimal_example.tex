    %% The MIT License (MIT)
%%
%% Copyright (c) 2015 Daniil Belyakov
%%
%% Permission is hereby granted, free of charge, to any person obtaining a copy
%% of this software and associated documentation files (the "Software"), to deal
%% in the Software without restriction, including without limitation the rights
%% to use, copy, modify, merge, publish, distribute, sublicense, and/or sell
%% copies of the Software, and to permit persons to whom the Software is
%% furnished to do so, subject to the following conditions:
%%
%% The above copyright notice and this permission notice shall be included in all
%% copies or substantial portions of the Software.
%%
%% THE SOFTWARE IS PROVIDED "AS IS", WITHOUT WARRANTY OF ANY KIND, EXPRESS OR
%% IMPLIED, INCLUDING BUT NOT LIMITED TO THE WARRANTIES OF MERCHANTABILITY,
%% FITNESS FOR A PARTICULAR PURPOSE AND NONINFRINGEMENT. IN NO EVENT SHALL THE
%% AUTHORS OR COPYRIGHT HOLDERS BE LIABLE FOR ANY CLAIM, DAMAGES OR OTHER
%% LIABILITY, WHETHER IN AN ACTION OF CONTRACT, TORT OR OTHERWISE, ARISING FROM,
%% OUT OF OR IN CONNECTION WITH THE SOFTWARE OR THE USE OR OTHER DEALINGS IN THE
%% SOFTWARE.

% The font could be set to Windows-specific Calibri by using the 'calibri' option
\documentclass[a4paper]{mcdowellcv}


% Set applicant's personal data for header
\name{Erika Musterfrau}
\address{Berlin, Germany \linebreak 
\link{https://github.com/ErikaMusterfrau}{GitHub}, \link{https://www.linkedin.com/in/erika_musterfrau/}{LinkedIn}}
\contacts{
+49 123 456 789 00 \linebreak 
\link{mailto:erika@musterfrau.com}{erika@musterfrau.com}}

\begin{document}
	
    \makeheader
    
    \begin{cvsection}{Section with Itemize}
        \begin{cvitemize}
            {Left}
            {Center heading of \texttt{CVItemize}}
            {Right}
                \cvitem{\texttt{CVItem} of arbitrary length. May be plain text or    
                      \begin{itemize}
                        \item sub-items
                        \item and one more.
                    \end{itemize}
                    }
                    [Remark]
                    [List of Keywords]
                \cvitem{Only content is mandatory.}[][]
                \cvitem{Further information without remark.}[][some nice skills]
        \end{cvitemize}
    \end{cvsection}

    \begin{cvsection}{Section with Subsections}
        \begin{cvsubsection}
            {Left}
            {Center heading of \texttt{CVSubsection}}
            {Right}
            {keywords}
                The main difference is that \texttt{CVSubsection} has a list of keywords for the entire section while \texttt{cvitem} has list of keywords per bullet point.
        \end{cvsubsection}
        %
        \begin{cvsubsection}
            {Only left is mandatory.}
            {}
            {}
            {}
        \end{cvsubsection}
        %
    \end{cvsection}

\end{document}
